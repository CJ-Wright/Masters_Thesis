\chapter{Atomic Structure: Extraction and Application}
\section{Atomistic Goals}
The only way to truly understand the fundamental source of material and chemical properties is through atomic structure.
The goal of atomistic engineering is to produce novel structures and combinations of structures to engender new properties and functions.
This includes producing stronger materials, more durable catalysts, more energy dense batteries, and many more engineering applications.
The true power of atomistic engineering has been shown in biochemistry and pharmaceutical design.
Although the production of drugs and biomedical treatments is usually considered to be rather far from the field of catalyst design and materials science, the atomistic nature of these fields can not be denied.
The field of protein structural analysis stands as an example of structural science, elucidating the three dimensional coordinates of thousands of atoms.
These structures are then used to describe how the molecular machinery of the biological world works, enabling the development of new drugs and treatments for diseases and a deeper understanding of how we evolved.
The development of protein inhibitor drugs, which are important to so many treatments, would have not been possible without very detailed atomic structures.
The aspiration of this work is to create this level of accuracy and utility, generating structures which allow for the understanding of how materials work on a fundamental level.
\section{Atomistic Experiments}
\subsection{Single Crystal Diffraction}
\subsection{Electron Microscopy}
\subsection{X-ray Total Scattering}

\section{Atomistic Simulations}
The goals of atomistic simulations are usually to produce atomic structures from quantum mechanical first principles, as in the case of Density Functional Theory (DFT), or classical approximations to quantum mechanics.
\subsection{Density Functional Theory}
\subsection{Classical Force Field}
\subsection{Monte Carlo and Statistical Mechanics}

Maybe put the ensemble and PES work here, since it is more general than the PDF per say. Also the rational for the gradients and fast computation make much more sense knowing we are going to be very sample happy and follow the gradient of the PES.