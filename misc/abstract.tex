Engineering the next generation of materials, especially nanomaterials, requires a detailed understanding of the material's underlying atomic structure.
These structures give us better insight into structure-property relationships, allowing for property driven material design on the atomic level.
Even more importantly, understanding structures in-situ will translate stimuli and responses on the macroscopic scale to changes on the nanoscale.
Despite the importance of precise atomic structures for materials design, solving atomic structures is difficult both experimentally and computationally.
Atomic pair distribution functions (PDFs) provide information on atomic structure, but the difficulty of extracting the PDF from x-ray total scattering measurements limits their use.
Translating the PDF into an atomic structure requires the search of a very high dimensional space, the set of all potential atomic configurations.
The large computational cost of running these simulations also limits the use of PDF as an atomistic probe.

This work aims to address these issues by developing 1) novel statistical mechanical approaches to solving material structures, 2)  fast simulation of x-ray total scattering and atomic pair distribution functions (PDFs), and 3) data processing procedures for experimental x-ray total scattering measurements.
First, experimentally derived potential energy surfaces (PES) and the statistical mechanical ensembles used to search them are developed.
Then the mathematical and computational framework for the PDF and its gradients will be discussed.
The combined PDF-PES-ensemble system will be benchmarked against a series of nanoparticle structures to ascertain the efficiency and effectiveness of the system.
Experimental data processing procedures, which maximize the usable data, will be presented.
Finally, preliminary results from experimental x-ray total scattering measurements will be discussed.
This work presents one of the most complete end-to-end systems for processing and modeling x-ray total scattering PDF data, potentially allowing for high-throughput structural solution.