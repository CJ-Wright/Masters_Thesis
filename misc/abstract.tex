Engineering the next generation of materials, especially nanomaterials, requires a detailed understanding of the material's underlying atomic structure.
 Understanding these structures we can obtain a better understanding of the structure-property relationship, allowing for a property driven rational design of the material structure on the atomic level.
 Even more importantly, understanding the structure in-situ will allow the reversal of the flow of information, translating stimuli and responses on the macroscopic system level to changes in the atomic structure.
 Despite the importance of atomic structures for designing materials, solving the atomic structure of materials is difficult both experimentally and computationally.
 Atomic pair distribution functions (PDFs) have been shown to provide information on atomic structure, although extracting the PDF from x-ray total scattering measurements can be difficult.
 Computationally, translating the PDF to an atomic structure requires the search of a very high dimensional space and can be computationally expensive.

This work aims to address these issues by developing 1) novel statistical mechanical approaches to solving material structures, 2)  fast simulation of x-ray total scattering and atomic pair distribution functions (PDFs), and 3) data processing procedures for experimental x-ray total scattering measurements.
First, experimentally derived potential energy surfaces (PES) and the statistical mechanical ensembles used to search them are developed.
Then the mathematical and computational framework for the PDF and its gradients will be discussed.
The combined PDF-PES-ensemble system will be benchmarked against a series of nanoparticle structures to ascertain the efficiency and effectiveness of the system.
Experimental data processing procedures, which maximize the usable data, will be presented.
Finally, preliminary results from experimental x-ray total scattering measurements will be discussed.
This work presents one of the most complete end-to-end systems for processing and modeling x-ray total scattering PDF data, potentially allowing for high-throughput structural solution.