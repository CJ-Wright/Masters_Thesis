\documentclass{uscthesis}

%%%%%%%%%%%%%%%%%%%%%%%%%%%%%%%%%%%%%%%%%%
%%% Options include: [forbinding], which produces 
%%% an alternative title page and an appropriate
%%% binding margin,  [honors] for Honors College theses,
%%% and [durt] for undergraduate thesis submitted as part
%%% part of the distinction in mathematics program.
%%%%%%%%%%%%%%%%%%%%%%%%%%%%%%%%%%%%%%%%%%

%%%%%%%%%%%%%%%%%%%%%%%%%%%%%%%%%%%%%%%%%%
%%  LaTeX Preamble
%%%%%%%%%%%%%%%%%%%%%%%%%%%%%%%%%%%%%%%%%%
\usepackage{graphicx}
\usepackage{amsmath,amsfonts,amssymb}
\usepackage[version=3]{mhchem} % Formula subscripts using \ce{}




%%%%%%%%%%%%%%%%%%%%%%%%%%%%%%%%%%%%%%%%%%
%% You should include above
%% any LaTeX packages that you need.  Most packages should work 
%% with this documentclass.
%%%%%%%%%%%%%%%%%%%%%%%%%%%%%%%%%%%%%%%%%

\usepackage[style=uscauthoryear, backend=biber]{biblatex}
\bibliography{references}


%%%%%%%%%%%%%%%%%%%%%%%%%%%%%%%%%%%%%%%%
%% The lines above specify a BibTeX style which controls 
%% the appearance of the bibliography and how citations to
%% the bibliography within the text will work.  It is based on the biblatex.sty
%% package and provides a Chicago style, as preferred by the Graduate School.
%% There are other acceptable styles.  Indeed, different academic disciplines
%% have different styles.
%% 
%% The line  \bibliography{references} will cause LaTeX is search for a file
%% called references.bib.  This file could be named differently.  For example
%% \bibliography{henry} would provoke a search for henry.bib.  The
%% file reference.bib (or henry.bib) is one you will have to produce.  It is
%% a BibTeX database of references you use.
%% 
%% There are a number of alternate ways to address your bibliographic needs.
%% See the documentation uscthesisdoc.pdf  for a discussion of the different options.
%%
%%
%% 
%%In any case, this  is a good spot to ask LaTeX to load what it needs to handle
%% literature citations and to layout the bibliography. 
%%
%%%%%%%%%%%%%%%%%%%%%%%%%%%%%%%%%%%%%%%%%%


\newtheorem{thm}{Theorem}[chapter]
\newtheorem*{thmun}{Theorem}
\newtheorem{cor}[thm]{Corollary}
\newtheorem{lem}[thm]{Lemma}
\theoremstyle{definition}
\newtheorem{defn}[thm]{Definition}
\newtheorem{ex}[thm]{Example}
\theoremstyle{plain}


%%%%%%%%%%%%%%%%%%%%%%%%%%%%%%%%%%%%%%%%%%%%
%%  These are just a few sample lines. Put here any 
%%  commands of your own devising that you want to use.
%%  If these examples are no use to you, omit them.
%%%%%%%%%%%%%%%%%%%%%%%%%%%%%%%%%%%%%%%%%%%%%

\newcommand{\grad}[1]{\vec{\nabla} #1} % for gradient

%%%%%%%%%%%%%%%%%%%%%%%%%%%%%%%%%%%%%%%%%%%%%%%%%%%%%%
%%             The Front Matter
%%  The section below deals with the material that comes 
%%  before the actual content of the document: The title 
%%  page, abstract, acknowledgments,etc.
%%
%%  Some of it is required.
%%%%%%%%%%%%%%%%%%%%%%%%%%%%%%%%%%%%%%%%%%%%%%%%%%%%%%

\title{Solving Atomic Structure using Statistical Mechanical Searches on X-ray scattering derived Potential Energy Surfaces}

\author{Christopher James}{Wright}    %% First Name then 
                                 %% Last Name

\date{2016}                      %% The year of graduation

\otherdegrees{
Bachelor of Science\\
Brown University 2014\\ [\baselineskip]
%% The \\ on this line is 
}                                %% ESSENTIAL!

\degree{Masters of Science}     %% The Graduate School provides 
                                 %% a list of official degrees.
\field{Chemical Engineering}              %% Fields also provided by the 
                                 %% Graduate School.
\college{College of Engineering and Computing}  %%As listed by Grad School

\advisor {Dr.}{Xiao-Dong Zhou}{Major Professor}  %%% Be sure the 
\readera{Dr.}{Thomas Vogt}{Committee Member}     %%% third field is 
\readerb{Dr.}{Mark Uline}{Committee Member}          %%% the one used in 
\readerc{Dr.}{Jochen Lauterbach}{Committee Member} %%% your department.
%%% If you have just two readers, for example, leave out \readerc and
%%% \readerd
%%%
%%% For Honors College theses use \reader{}{}   NO third field.
%%% The commands \otherdegrees, \degree, \field, \college, \readera, etc.
%%% are not used under the honors option.
%%%%%%%%%%%%%%%%%%%%%%%%%%%%%%%%%%%%%%%%%%%%%%%%%%%%%%%

\dean{Lacy Ford}   %% The Dean of the Graduate School
                   %% BE SURE TO CHECK THE NAME OF THE
                   %%PERSON CURRENTLY HOLDING THIS POSITION
                     %% For Honors College theses use
                     %% \schcsigner{}{}.  For example,
                     %% \schcsigner{Dr.}{Davis Baird}

\copyrightpage       %% This is optional. It makes a 
                     %% copyright page that will appear 
                     %% immediately after the title page.

\abstract{abstract}  %% This calls the file herkimer.tex but 
                     %% but you might replace herkimer by 
                     %% anything you like, for example by 
                     %% abstract. Note, the Graduate School
                     %% REQUIRES that PhD dissertations have 
                     %% abstracts.
                     %%
                     %% For Honors College theses use
                     %% \honorsabstract{}


\acknowledgments{thanks} %% This calls the file thanks.tex 
%% This is optional       %% where you have put your 
                          %%acknowledgments.

\dedication{dedication}   %% Calls dedication.tex
%%% Also optional

% \preface{forward}    %% Calls forward.tex.  Optional.

\makeLoT               %% Issue this command if your work has 
                       %% four or more tables.  A list of tables 
                       %% will be produced automatically.

\makeLoF               %% works the same way but for figures.

%%%%%%%%%%%%%%%%%%%%%%%%%%%%%%%%%%%%%%%%%%%%%%%%%%%%%%%%%%%%
%%  Finally, here is the meat.  The idea is to compose a 
%%  .tex file for each section of your thesis or dissertation.  
%%  Then use LaTeX's \include command to put them all together.  
%%  Doing it this way makes it easier to change the order of 
%%  exposition as your writing is in progress.  Also it
%%  makes it easy to print out just one section. The \include
%%  command always starts a new page. So every section would 
%%  start on a new page.  If you would like for sections just
%%  to continue, after the appropriate vertical space, on the
%%  current page, then use the \input command instead of the 
%%  \include command.
%%%%%%%%%%%%%%%%%%%%%%%%%%%%%%%%%%%%%%%%%%%%%%%%%%%%%%%%%%%%

\begin{document}

\chapter*{Introduction}
This is the introduction to the thesis.
    %% Calls Introduction.tex
                          %% Honors theses are required to 
                          %% have an Introduction.  For
                          %% Honors theses, the file 
                          %% Introduction.tex should begin
                          %%
                          %% \chapter*{Introduction}
                          %% followed by the text of the 
                          %% introduction.


\chapter{The importance of atomistic structure}
         %% The three sections of Chapter 1 
\section{Atomic Structure Experiments}
\subsection{TEM/STEM}
\subsubsection{In-Situ Experiments}

\subsection{X-ray Total Scattering}
\subsubsection{In-Situ Experiments}

             %% are in the files Squares.tex,  
\section{Computationally Extracting Structure from Measurements}

\subsection{Small box}
\subsection{Large box}
\subsection{Exotic Simulations}
	   %% Cubes.tex and Hypercubes.tex
       
                           
\chapter{Statistical Mechanical Potential Energy Surface Minima Search}
The general strategy of my work is to use statistical mechanical ensembles to search potential energy surfaces (PESs) which include information about the atomic structure. This information will come from experimental data or computational sources like Density Functional Theory or Embedded Atom Method calculations.


\chapter{Potential Energy Surfaces}


% \include{PDF_math_comp}
% \include{math}
% \include{computation}

% \include{Benchmarking}


%\include{Conclusion}     %% Honors theses are required to 
                          %% have an unnumbered chapter
                          %% for conclusions.  The file
                          %% Conclusion.tex should begin
                          %%   
                          %% \chapter*{Conclusion}
                          %% followed by the appropriate
                          %% text.

% \printbibliography %%  This is the command to use to
			       %%  insert the bibliography if you are using
                           %% the biblatex.sty package.  See the 
                           %% uscthesisdoc.pdf documentation for
                           %% for alternative bibliographic systems.     

\Appendix                 %% Use this command if you have one 
                          %% appendix. Use \Appendices if you 
                          %% have more than one.
	
% \input{toolong}         %% Calls toolong.tex which contains
                          %% an appendix. After issuing the 
                        %% command \Appendix or \Appendices
                        %% you must use \input not \include
                        %% to load the first appendix.

\end{document}
%%%%%%%%%%%%%%%%%%%%%%%%%%%%%%%%%%%%%%%%%%%%%%%%%%%%%%%%%%%%%%%
