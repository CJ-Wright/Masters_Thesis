\documentclass{beamer}
\usepackage[latin1]{inputenc}
\usepackage{beamerthemesplit}
\usepackage{graphics,epsfig, subfigure}
\usepackage{epstopdf}
\usepackage{url}

\definecolor{scgarnet}{RGB}{125,9,0}
\setbeamercovered{transparent}
\mode<presentation>
{  \usetheme{Madrid}
  \usecolortheme[named=scgarnet]{structure}
  \useinnertheme{circles}
  \usefonttheme[onlymath]{serif}
  \setbeamercovered{transparent}
  \setbeamertemplate{blocks}[rounded][shadow=true]
  \setbeamertemplate{footline}{	
		\begin{center}
		\includegraphics[width=.7in]{template_slides/SCLogo}~~~
      	\end{center} 
      	}
}
\title{Solving Atomic Structures using Statistical Mechanical Searches on X-ray Scattering Derived Potential Energy Surfaces}
\author{Christopher J. Wright}
\institute{Department of Chemical Engineering\\ University of South Carolina\\
Brookhaven National Laboratory\\ Photon Sciences Directorate}
\date{\today}
\begin{document}

\begin{frame}
\titlepage
\end{frame}

\section{Introduction}
\subsection{Goals}
\begin{frame}
  \frametitle{Atomistic Engineering}
\end{frame}

\subsection{Getting to the goals}

\section{Searches}
\subsection{Potential Energy Surfaces}
\subsection{Ensembles}

\section{PDF Theory and Computation}
\subsection{PDF Math}
\begin{frame}
  \frametitle{Elastic X-ray Scattering and the PDF}
  
\end{frame}
\subsection{PDF computation}

\section{Benchmarking}

\section{Experimental Data Processing}

\section{Preliminary Experimental Results}



\end{document}
