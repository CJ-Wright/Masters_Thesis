\chapter{Statistical Mechanical Ensembles and Potential Energy Surfaces}
\section{Introduction}
The approach taken in this work for solving the atomic structures of materials is one of optimization.
The positional variables of the system are optimized so as to minimize the value of a potential energy surface (PES).
The 
\section{Potential Energy Surfaces}
A PES simply describes the potential energy of the system as a function of all its relevent coordinates in phase space, essentially providing a mapping $\mathbb{R}^{n} \rightarrow \mathbb{R}$.
Usually these coordinates are the positions of the atoms $q$ and their conjugate the momenta $p$.
Note that there could be more variables associated with the system, for instance the magnetic moments of the atoms could play a role in describing the system.
In this magnetic system there would be positional variables for the atomwise spin vectors and their "momenta".
Application of the term "momenta" might seem odd here, as the magnetic spin does not have a mass or a velocity.
However, since the magnetic "position" is defined on the PES we need to describe its conjugate varible to properly formulate Hamitonian dynamics and the kinetic portion of the PES.

\subsection{Experimentally Derived Potential Energy Surfaces}
Generally PESs are obtained from purely computational experiments including: ab-initio DFT, classical approximations via the embedded atom method, or even parameter driven models with experimentally fitted parameters.
However, one can dervive a PES from an experiment which describes how well the model reproduces the experimental data.
In this case one needs a theoretical and computational framework mapping the atomistic variables of the simulation to the same space of the data obtained from the experiment.
This allows the experiment to be compared directly against the predicted data via an experimentally derived PES.
\subsubsection{Potentials}
For an experiment which produces 1D data, like powder diffraction, EXAFS or XPS, the implemented potentials are:
\begin{equation} \label{chi}
\chi^{2} = 
\sum_{a=a_\mathrm{min}}^{a_\mathrm{max}} \left(A_{_\mathrm{obs}} - \alpha A_{_\mathrm{calc}}\right)^{2}
\end{equation}
\begin{equation}\label{Rw}
Rw = 
\sqrt{\frac{\sum_{a=a_\mathrm{min}}^{a_\mathrm{max}} \left(A_{_\mathrm{obs}} - \alpha A_{_\mathrm{calc}}\right)^{2}}{\sum_{a=a_\mathrm{min}}^{a_\mathrm{max}} A_{_\mathrm{obs}}^{2}}}
\end{equation}
\begin{equation} \label{alpha}
\alpha  = \frac{\sum_{a=a_\mathrm{min}}^{a_\mathrm{max}}A_\mathrm{_\mathrm{obs}}A_{_\mathrm{calc}}}{\sum_{a=a_\mathrm{min}}^{a_\mathrm{max}} A_{_\mathrm{calc}}^{2}} = \frac{\vec{A}_{_\mathrm{obs}}\cdot\vec{A}_\mathrm{calc}}{|\vec{A}_\mathrm{calc}|^{2}}
\end{equation}
where $A_{calc}$ and $A_{obs}$ are the calculated and observed 1D experimental data.  Note that $A_{calc}$ has a dependence on $q$, the positions of the system.

\subsubsection{Forces}
\begin{equation}
\grad{\chi^{2}} =
- 2 \sum_{a=a_\mathrm{min}}^{a_\mathrm{max}} (\alpha \frac{\partial A_{_\mathrm{calc}}}{\partial q_{i, w}} + A_{_\mathrm{calc}} \frac{\partial \alpha}{\partial q_{i, w}} ) (A_{_\mathrm{obs}} - \alpha A_{_\mathrm{calc}})
\end{equation}
\begin{equation}
\grad{Rw} = 
\frac{Rw}{\chi^{2}} \sum_{a=a_\mathrm{min}}^{a_\mathrm{max}} (\alpha \frac{\partial A_{_\mathrm{calc}}}{\partial q_{i, w}} + A_{_\mathrm{calc}} \frac{\partial \alpha}{\partial q_{i, w}} ) (\alpha A_{_\mathrm{calc}}  - (A_{_\mathrm{obs}}))
\end{equation}
\begin{equation}
\frac{\partial \alpha}{\partial q_{i, w}}  = 
\frac{(\sum_{a=a_\mathrm{min}}^{a_\mathrm{max}} A_{_\mathrm{obs}} \frac{\partial A_{_\mathrm{calc}}}{\partial q_{i, w}}- 2\alpha \sum_{a=a_\mathrm{min}}^{a_\mathrm{max}} A_{_\mathrm{calc}} \frac{\partial A_{_\mathrm{calc}}}{\partial q_{i, w}})}{\sum_{a=a_\mathrm{min}}^{a_\mathrm{max}} A_{_\mathrm{calc}}^{2}}
\end{equation}
INCLUDE INVERT. DISCUSS INVERT A BUNCH. ALSO COMPARE RW AND CHI**2, POTENTIALY WITH A FIGURE.

\section{Ensembles}
While PESs describe which atomic configurations are the most desirable and how the atoms would like to get there, the ensemble describes how the atoms move on the PES.
The abstraction of the PES from the ensemble is an important one, as it allows for the reuse and exchange of both PESs and ensembles for a wide array of problems.
Statistical mechanical ensembles can be described in two ways, analyticly and stochasticly.
For long simulation times and fine enough numerical or analytical integration these two descriptions should be identical.
In either case one starts by defining the Hamiltonian $\mathcal{H}$ as the total energy of the system.
Thus, the Hamiltonian is described as the sum of the potential $U(q)$ and kinetic $K(p)$ energies, where $q$ is the positions of the atoms and $p$ is their momenta
\begin{equation} \label{Hamiltonian}
  \mathcal{H}(q, p) = U(q) + K(p)
\end{equation}
\noindent where $K(p) = \frac{1}{2}\sum_{i} \frac{p_{i}^{2}}{m_{i}}$ and $i$ denotes the $i$th particle.
Analyticly one generally defines a partition function, which describes the sum of probabilities over all potential atomic states.
\[
\Xi = \sum_{i} P_{i}(q, p)
\]
where $P_{i}$ is the probability of the $i$th state and is a function of the total energy of that state.
This partition function can then be used to obtain the probabilty of any specific state.
\subsection{Hamiltonian Monte Carlo}
\subsubsection{No-U-Turn-Sampling}
\subsection{Grand Canonical Monte Carlo}
\subsubsection{Configuraitonal Biasing}
