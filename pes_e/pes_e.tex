\chapter{Statistical Mechanical Ensembles and Potential Energy Surfaces}
\section{Introduction}
The approach taken in this work for solving the atomic structures of materials is one of optimization.
The positional variables of the system are optimized so as to minimize the value of a potential energy surface (PES).
The 
\section{Potential Energy Surfaces}
A PES simply describes the potential energy of the system as a function of all its relevent coordinates in phase space, essentially providing a mapping $\mathbb{R}^{n} \rightarrow \mathbb{R}$.
Usually these coordinates are the positions of the atoms $q$ and their conjugate the momenta $p$.
Note that there could be more variables associated with the system, for instance the magnetic moments of the atoms could play a role in describing the system.
In this magnetic system there would be positional variables for the atomwise spin vectors and their "momenta".
Application of the term "momenta" might seem odd here, as the magnetic spin does not have a mass or a velocity.
However, since the magnetic "position" is defined on the PES we need to describe its conjugate varible to properly formulate Hamitonian dynamics and the kinetic portion of the PES.

\subsection{Experimentally Derived Potential Energy Surfaces}
Generally PESs are obtained from purely computational experiments including: ab-initio DFT, classical approximations via the embedded atom method, or even parameter driven models with experimentally fitted parameters.
However, one can dervive a PES from an experiment which describes how well the model reproduces the experimental data.
In this case one needs a theoretical and computational framework mapping the atomistic variables of the simulation to the same space of the data obtained from the experiment.
This allows the experiment to be compared directly against the predicted data via an experimentally derived PES.
\subsubsection{Potentials}
For an experiment which produces 1D data, like powder diffraction, EXAFS or XPS, the implemented potentials are:
\begin{equation} \label{chi}
\chi^{2} = 
\sum_{a=a_\mathrm{min}}^{a_\mathrm{max}} \left(A_{_\mathrm{obs}} - \alpha A_{_\mathrm{calc}}\right)^{2}
\end{equation}
\begin{equation}\label{Rw}
Rw = 
\sqrt{\frac{\sum_{a=a_\mathrm{min}}^{a_\mathrm{max}} \left(A_{_\mathrm{obs}} - \alpha A_{_\mathrm{calc}}\right)^{2}}{\sum_{a=a_\mathrm{min}}^{a_\mathrm{max}} A_{_\mathrm{obs}}^{2}}}
\end{equation}
\begin{equation}\label{INVERT}
  \chi^{2}_{\mathrm{INVERT}} = \frac{1}{N}\sum_{j}\sum_{r}[A_{obs}(r) - \alpha A_{j_\mathrm{calc}}(r)]^{2}
\end{equation}
\begin{equation} \label{alpha}
\alpha  = \frac{\sum_{a=a_\mathrm{min}}^{a_\mathrm{max}}A_\mathrm{_\mathrm{obs}}A_{_\mathrm{calc}}}{\sum_{a=a_\mathrm{min}}^{a_\mathrm{max}} A_{_\mathrm{calc}}^{2}} = \frac{\vec{A}_{_\mathrm{obs}}\cdot\vec{A}_\mathrm{calc}}{|\vec{A}_\mathrm{calc}|^{2}}
\end{equation}
where $A_{calc}$ and $A_{obs}$ are the calculated and observed 1D experimental data
and $A_{calc, j}$ is the calculated data for a single atom interacting with the other atoms of the system. 
Note that $A_{calc}$ has a dependence on $q$, the positions of the system.

\subsubsection{Forces}
\begin{equation}
\grad{\chi^{2}} =
- 2 \sum_{a=a_\mathrm{min}}^{a_\mathrm{max}} (\alpha \frac{\partial A_{_\mathrm{calc}}}{\partial \gamma_{i, w}} + A_{_\mathrm{calc}} \frac{\partial \alpha}{\partial \gamma_{i, w}} ) (A_{_\mathrm{obs}} - \alpha A_{_\mathrm{calc}})
\end{equation}
\begin{equation}
\grad{Rw} = 
\frac{Rw}{\chi^{2}} \sum_{a=a_\mathrm{min}}^{a_\mathrm{max}} (\alpha \frac{\partial A_{_\mathrm{calc}}}{\partial \gamma_{i, w}} + A_{_\mathrm{calc}} \frac{\partial \alpha}{\partial \gamma_{i, w}} ) (\alpha A_{_\mathrm{calc}}  - (A_{_\mathrm{obs}}))
\end{equation}
\begin{equation}
\frac{\partial \alpha}{\partial \gamma_{i, w}}  = 
\frac{(\sum_{a=a_\mathrm{min}}^{a_\mathrm{max}} A_{_\mathrm{obs}} \frac{\partial A_{_\mathrm{calc}}}{\partial \gamma_{i, w}}- 2\alpha \sum_{a=a_\mathrm{min}}^{a_\mathrm{max}} A_{_\mathrm{calc}} \frac{\partial A_{_\mathrm{calc}}}{\partial \gamma_{i, w}})}{\sum_{a=a_\mathrm{min}}^{a_\mathrm{max}} A_{_\mathrm{calc}}^{2}}
\end{equation}
\begin{equation}
  \grad{\chi^{2}_{\mathrm{INVERT}}} = \frac{-2}{N} \sum_{a=a_\mathrm{min}}^{a_\mathrm{max}} \sum_{j}(\alpha \frac{\partial A_{j_\mathrm{calc}}}{\partial \gamma_{i, w}} + A_{j_\mathrm{calc}} \frac{\partial \alpha}{\partial \gamma_{i, w}} ) (A_{_\mathrm{obs}} - \alpha A_{j_\mathrm{calc}})
\end{equation}
where $\gamma_{i, w}$ is the $i$th arbitrary positional variable in the $w$th direction.
The concept of an "arbitrary positional variable" might seem a bit cumbersome but it allows us to define the forces for any atomic parameter which can be represented as a vector in 3-space.
This comes in handy when trying to define the forces acting on variables like anistropic displacement parameters or atomic magnetic spins.

DISCUSS INVERT A BUNCH. ALSO COMPARE RW AND CHI**2, POTENTIALY WITH A FIGURE.

\section{Ensembles}
While PESs describe which atomic configurations are the most desirable and how the atoms would like to get there, the ensemble describes how the atoms move on the PES.
The abstraction of the PES from the ensemble is an important one, as it allows for the reuse and exchange of both PESs and ensembles for a wide array of problems.
Statistical mechanical ensembles can be described in two ways, analyticly and stochasticly.
For long simulation times and fine enough numerical or analytical integration these two descriptions should be identical.
In either case one starts by defining the Hamiltonian $\mathcal{H}$ as the total energy of the system.
Thus, the Hamiltonian is described as the sum of the potential $U(q)$ and kinetic $K(p)$ energies, where $q$ is the positions of the atoms and $p$ is their momenta
\begin{equation} \label{Hamiltonian}
  \mathcal{H}(q, p) = U(q) + K(p)
\end{equation}
\noindent where $K(p) = \frac{1}{2}\sum_{i} \frac{p_{i}^{2}}{m_{i}}$ and $i$ denotes the $i$th particle.
Analyticly one generally defines a partition function, which describes the sum of probabilities over all potential atomic states.
\[
\Xi = \sum_{i} P_{i}(q, p)
\]
where $P_{i}$ is the probability of the $i$th state and is a function of the total energy of that state.
This partition function can then be used to obtain the probabilty of any specific state.
\subsection{Hamiltonian Monte Carlo}
In order to model dynamics we need to describe the motion of the particles in our system, thus:
\begin{align}
  \frac{d q_{i}}{d t} &= \frac{\partial \mathcal{H}}{\partial p_{i}} = p_{i}\\
  \frac{d p_{i}}{d t} &= -\frac{\partial \mathcal{H}}{\partial q_{i}} = -\grad{U}  
\end{align}
Using these equations we can derive the position and momentum vectors at any point in time using the leap-frog algorithm:
\begin{align}
  p_{i}(t + \delta t/2) &= p_{i}(t) - \frac{\delta t}{2} \frac{\partial }{\partial q_{i}}U(q(t))\\
  q_{i}(t+\delta t)  &= q_{i}(t) + \delta t * p_{i}(t+\delta t/2)\\
  p_{i}(t+\delta t) &=   p_{i}(t + \delta t/2)- \frac{\delta t}{2} \frac{\partial }{\partial q_{i}}U(q(t+\delta t))
\end{align}
Note that $\frac{\partial}{\partial q_{i}}$ is the gradient with respect to $q$ where $i$ denotes the $i$th atom being moved.
Using this notation the gradient is
\begin{equation} \label{grad}
  \grad{U} = \begin{bmatrix}
    \frac{\partial U}{\partial q_{0, x}} & \frac{\partial U}{\partial q_{0, y}} & \frac{\partial U}{\partial q_{0, z}} \\
    \vdots & \frac{\partial U}{\partial q_{i, w}} & \vdots \\
    \frac{\partial U}{\partial q_{n, x}} & \frac{\partial U}{\partial q_{n, y}} & \frac{\partial U}{\partial q_{n, z}} \\
  \end{bmatrix} = 
  \begin{bmatrix}
    \vec{\mathcal{F}}_{0} \\
    \vdots \\
    \vec{\mathcal{F}}_{i} \\
    \vdots \\
    \vec{\mathcal{F}}_{n}\\
  \end{bmatrix}
\end{equation}
where $\frac{\partial}{\partial q_{i, w}}$ is the derivative with respect to $q$ where $w$ denotes direction of the derivative ($x$, $y$, or $z$),  $n$ is the number of atoms and $U$ is the potential which depends on $q$, and $\vec{\mathcal{F}_{i}}$ is the "force" on the $i$th atom.
\subsubsection{No-U-Turn-Sampling}
\subsection{Grand Canonical Ensemble}
\subsubsection{Ensemble description}
In the Grand Canonical Ensemble (GCE) two sets of variables are allowed to change, the atomic positions and the total number of atoms and their associated identities.  
These two variables are controlled by temperature and chemical potential.  
The partition function is
\begin{equation}
  \Xi = e^{-\beta(\mathcal{H} +\mu)}
\end{equation}
This is translated into a Monte Carlo system, producing Grand Canonical Monte Carlo (GCMC).
\subsubsection{Grand Canonical Monte Carlo}
While the probabilities for atomic motion are the same as in the Canonical Ensemble, the addition or removal of an atom have their own probabilities. For the addition of an atom the probability is formally:
\begin{equation}
  \min[1, \frac{V}{(N+1) \Lambda(T)^{3}}e^{-\beta\Delta U + \beta \mu}]
\end{equation}
Similarly the removal of an atom has the probability:
\begin{equation}
  \min[1, \frac{(N)\Lambda(T)^{3}}{V}e^{-\beta\Delta U - \beta \mu}]
\end{equation}
However, both of these equations depend of the overall simulation volume and the thermal wavelength, which is undesirable as these are not really properties that we are of interest to these simulations.
Thus, we roll them into the definition of the chemical potential, essentially setting the base chemical potential to counteract these effects.
This makes certain that our simulation does not change if we change the overall cell volume.
A GCMC move consists of creating a new atomic configuration, where an atom has been added or removed, and checking the above criteria.
However, previous results have shown that this method is computationally expensive in dense liquids, and exceedingly expensive in solid materials.
The long simulation times are due to the random nature of the atomic additions or removals which produce: over-tightly packed atoms, atoms in the middle of nowhere, or unphysical vacancies.  
These configurations are rejected by the GCMC criteria but their probability of being sampled is much higher than configurations which are lower in energy, since the number of incorrect ways to add/remove atoms is much larger than the correct ways.
Thus we have implemented methods for biasing the atomic addition positions and the atomic removals toward configurations which are more likely to be accepted.
\subsubsection{GCMC biasing}
The first method is to remove some of the excess options from the probability pool.
Initially the insertion positions are calculated at random using a random number generator and scaled to the size of the simulation cell.
This produces probabilities which have floating point level precision, which is effectively infinite.
While this produces a potentially infinite number of ways to create energetically favorable configurations, the infinite ways to produce bad configurations is much larger.
Thus we can limit this by moving to voxels.
In this case atoms are added to the center of voxels which have a pre-set resolution, limiting our total number of valid addition points.
While this could produce some problems with ergodicity, we avoid this by allowing the atoms to translate throughout the system.
Each voxel has a probability of being tried:
\begin{equation}
  P_{i, j, k} = \frac{x y z}{a b c}
\end{equation}
where $x, y, z$ and $a, b, c$ are the resolutions and cell side lengths in the cardinal directions, respectively.
While this does help to limit the total probability space it does not tell us which voxels are likely to lead to better configurations, leading to many rejected atomic additions.
To combat this issue we can weigh the individual voxels, giving more probability to voxels which show promise and less to those with less likelihood to be accepted.

The approach most likely to yield success would be to measure the change in potential energy associated with the addition of an atom at the center of the voxel where the probability of a voxel to be tried is:
\begin{equation}
  P_{i, j, k} = \frac{e^{\beta \Delta U_{i, j, k}}}{\sum_{i, j, k}e^{\beta \Delta U_{i, j, k}}}
\end{equation}
where $\Delta U_{i, j, k}$ is the change in energy.
However, calculating $\Delta U_{i, j, k}$ can be particularly expensive, especially when calculating scattering from atomic positions.
The computational expense can be mitigated by using a cheaper potential, if only for the evaluation of the voxel energy, as previously shown.
Similar to previous work we can use the Lennard Jones potential to approximate the addition potential.
