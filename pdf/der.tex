\subsection{Analytical Gradients}
Many optimization algorithms and simulations methodologies, including HMC, require not only the potential energy of a given configuration but also the forces acting on that configuration.
These forces are described by the gradient of potential energy of the system which in turn requires the gradient of the PDF.
As previously shown the PDF is the Fourier Transform of the Debye equation.
Since the Fourier Transform is expressed as an integral we can exchange the order of the gradient and the integral, allowing us to calculate the analytical gradient of the Debye equation and FFT the resulting function.
The Debye equation, with a Debye-Waller vibrational correction is
\begin{equation}
F(Q) = \frac{1}{N \langle f \rangle^{2}} \sum_{j\neq i} f_i^{*}(Q)f_j(Q) \exp(-\frac{1}{2}\sigma_{ij}^{2}Q^{2}) \frac{\sin(Qr_{ij})}{r_{ij}}
\end{equation}
where
\begin{eqnarray}
  \sigma_{ij}^{2} = (\vec{u}_{ij} * \hat{d}_{ij})^{2}\\
  \vec{u}_{ij} = \vec{u}_{i} - \vec{u}_{j}\\
  \hat{d}_{ij} = \frac{\vec{d}_{ij}}{r_{ij}}\\
  r_{ij} = ||\vec{d}_{ij}|| \\
  \vec{d}_{ij} =
  \begin{bmatrix}
    q_{ix} - q_{jx}\\
    q_{iy} - q_{jy} \\
    q_{iz} - q_{jz}
  \end{bmatrix}
\end{eqnarray}
where $Q$ is the scatter vector, $f_i$ is atomic scattering factor of the $i$th atom, and $r_{ij}$ is the distance between atoms $i$ and $j$ and has $q$ dependence.
For simplicities sake we will break up $F(Q)$ so that
\begin{equation}
F(Q) = \alpha \sum_{j\neq i} \beta_{ij} \uptau_{ij} \Omega_{ij}
\end{equation}
where
\begin{eqnarray}
  \alpha = \frac{1}{N \langle f \rangle^{2}} \\
  \beta_{ij} = f_i^{*}(Q)f_j(Q)\\
  \uptau_{ij} = \exp(-\frac{1}{2}\sigma_{ij}^{2}Q^{2})\\
  \Omega_{ij} = \frac{\sin(Qr_{ij})}{r_{ij}}
\end{eqnarray}

\noindent The derivatives are as follows:
\begin{equation}
\frac{\partial}{\partial q_{i,w}} F{ (Q )} = \alpha \sum_{j} \beta_{ij} (\frac{\partial \uptau_{ij}}{\partial q_{i,w}}  \Omega_{ij} + \uptau_{ij} \frac{\partial \Omega_{ij}}{\partial q_{i,w}})
\end{equation}
where
\begin{eqnarray}
  \frac{\partial \Omega_{ij}}{\partial q_{i,w}}  = \frac{Q\cos(Qr_{ij}) - \Omega_{ij}}{r_{ij}^{2}} (q_{i,w}-q_{j,w})\\
  \frac{\partial \uptau_{ij}}{\partial q_{i,w}} = \frac{\sigma_{ij}Q^{2} \uptau_{ij}}{r_{ij}^{3}}   ((q_{i,w} - q_{j,w}) \sigma_{ij}- ( u_{i,w} - u_{j,w})r_{ij}^{2})
\end{eqnarray}

Since $\vec{u}_{ij}$ is a variable as well, we need the derivative with respect to it as well.
Thus
\begin{eqnarray}
\frac{\partial}{\partial u_{i,w}} F{ (Q )} = \alpha \sum_{j} \beta_{ij} \frac{\partial \uptau_{ij}}{\partial u_{i,w}}  \Omega_{ij}
\frac{\partial \uptau_{ij}}{\partial u_{i,w}} = - \frac{\sigma_{ij}Q^{2} \uptau_{ij}}{r_{ij}}  (q_{i,w} - q_{j,w})
\end{eqnarray}
\subsubsection{Without ADPs}
Without ADPs the equations simplify down to
\begin{equation}
F(Q) = \frac{1}{N \langle f \rangle^{2}} \sum_{j\neq} f_i^{*}(Q)f_j(Q) \frac{\sin(Qr_{ij})}{r_{ij}}
\end{equation}
and
 \begin{equation}
\frac{\partial}{\partial q_{i,w}} F{ (Q )} = \alpha \sum_{j} \beta_{ij} \frac{\partial \Omega_{ij}}{\partial q_{i,w}}
\end{equation}
use of these equations, when ADPs are not appropriate (like at cyrogenic temperatures), greatly speeds up the computaiton.

\subsubsection{Periodic Boundary Conditions}
Periodic boundary conditions can be helpful when simulating extended solids or large nanoparticles. In this case all the non-crystallinity is contained within the simulation box and the box is repeated to create the longer distance peaks observed in the PDF. To perform this we can break up the Debye equation into two main parts, the part that describes the interatomic distances within the simulation box and those between boxes. Neglecting the thermal motion portion:
\begin{equation}
  F(Q) = \frac{1}{N \langle f \rangle^{2}}(\sum_{j\neq i} f_i^{*}(Q)f_j(Q) \frac{\sin(Qr_{ij})}{r_{ij}} + \sum_{i,j} f_i^{*}(Q)f_j(Q) \frac{\sin(QR_{ij})}{R_{ij}})
\end{equation}
where
\begin{eqnarray}
  R = |\vec{r} + \vec{u}|\\
  \vec{u} = \gamma_1*\vec{a} + \gamma_2*\vec{b} + \gamma_3*\vec{c}
\end{eqnarray}
\todo[inline]{Need to include PBC ADP math}
\todo[inline]{Also should include PBC gradients, although they are trivial, maybe?}