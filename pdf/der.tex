\section{Theory}
To properly understand the PDF and its limitations we need to derive its mathematics.
The PDF has been previously derived many times so it is not re-derived here.
This discussion of the PDF and its gradients use the notation of Farrow and Billinge. \cite{Farrow2009}
\subsection{Derivation}
Many of the above techniques require the gradient of the PES.
This in turn requires the gradient of the PDF to be derived.
Mathematically treating thermal vibrations will also be discussed in this section.
Systems which are truly extended materials, like powders with particle sizes larger than 10nm, are best formulated as systems with periodic boundaries.
Thus, the equations for a periodically bound PDF need to be developed as well, with their gradients.
\subsection{Analytically Gradients}
Many optimization algorithms and simulations methodologies, including HMC, require not only the potential energy of a given configuration but also the forces acting on that configuration.
These forces are described by the gradient of potential energy of the system which in turn requires the gradient of the PDF.
As previously shown the PDF is the Fourier Transform of the Debye equation.
Since the Fourier Transform is expressed as an integral we can exchange the order of the gradient and the integral, allowing us to calculate the analytical gradient of the Debye equation and Fast Fourier Transform (FFT) the resulting function.
The Debye equation, with a Debye-Waller vibrational correction is
\begin{equation}
F(Q) = \frac{1}{N \langle f \rangle^{2}} \sum_{j\neq i} f_i^{*}(Q)f_j(Q) \exp(-\frac{1}{2}\sigma_{ij}^{2}Q^{2}) \frac{\sin(Qr_{ij})}{r_{ij}}
\end{equation}
where
\begin{eqnarray}
  \sigma_{ij}^{2} = (\vec{u}_{ij} * \hat{d}_{ij})^{2}\\
  \vec{u}_{ij} = \vec{u}_{i} - \vec{u}_{j}\\
  \hat{d}_{ij} = \frac{\vec{d}_{ij}}{r_{ij}}\\
  r_{ij} = ||\vec{d}_{ij}|| \\
  \vec{d}_{ij} =
  \begin{bmatrix}
    q_{ix} - q_{jx}\\
    q_{iy} - q_{jy} \\
    q_{iz} - q_{jz}
  \end{bmatrix}
\end{eqnarray}
where $Q$ is the scatter vector, $f_i$ is atomic scattering factor of the $i$th atom, and $r_{ij}$ is the distance between atoms $i$ and $j$ and has $q$ dependence. \cite{Jeong2002}
\todo[inline]{state that the $\vec{u}$ are the ADPs}
For simplicity's sake we will break up $F(Q)$ so that
\begin{equation}
F(Q) = \alpha \sum_{j\neq i} \beta_{ij} \uptau_{ij} \Omega_{ij} \label{eq:abto}
\end{equation}
where
\begin{eqnarray}
  \alpha = \frac{1}{N \langle f \rangle^{2}} \label{eq:alpha} \\
  \beta_{ij} = f_i^{*}(Q)f_j(Q)\label{eq:beta}\\
  \uptau_{ij} = \exp(-\frac{1}{2}\sigma_{ij}^{2}Q^{2}) \label{eq:tau}\\
  \Omega_{ij} = \frac{\sin(Qr_{ij})}{r_{ij}} \label{eq:omega}
\end{eqnarray}

\noindent The derivatives are as follows:
\begin{equation}
\frac{\partial}{\partial q_{i,w}} F{ (Q )} = \alpha \sum_{j} \beta_{ij} (\frac{\partial \uptau_{ij}}{\partial q_{i,w}}  \Omega_{ij} + \uptau_{ij} \frac{\partial \Omega_{ij}}{\partial q_{i,w}}) \label{eq:grad_fq}
\end{equation}
where
\begin{eqnarray}
  \frac{\partial \Omega_{ij}}{\partial q_{i,w}}  = \frac{Q\cos(Qr_{ij}) - \Omega_{ij}}{r_{ij}^{2}} (q_{i,w}-q_{j,w})\\
  \frac{\partial \uptau_{ij}}{\partial q_{i,w}} = \frac{\sigma_{ij}Q^{2} \uptau_{ij}}{r_{ij}^{3}}   ((q_{i,w} - q_{j,w}) \sigma_{ij}- ( u_{i,w} - u_{j,w})r_{ij}^{2})
\end{eqnarray}

Since $\vec{u}_{ij}$ is a variable as well, we need the derivative with respect to it as well.
Thus
\begin{eqnarray}
\frac{\partial}{\partial u_{i,w}} F{ (Q )} = \alpha \sum_{j} \beta_{ij} \frac{\partial \uptau_{ij}}{\partial u_{i,w}}  \Omega_{ij}\\
\frac{\partial \uptau_{ij}}{\partial u_{i,w}} = - \frac{\sigma_{ij}Q^{2} \uptau_{ij}}{r_{ij}}  (q_{i,w} - q_{j,w})
\end{eqnarray}
\subsubsection{Without ADPs}
Without ADPs the equations simplify down to
\begin{equation}
F(Q) = \frac{1}{N \langle f \rangle^{2}} \sum_{j\neq} f_i^{*}(Q)f_j(Q) \frac{\sin(Qr_{ij})}{r_{ij}}
\end{equation}
and
 \begin{equation}
\frac{\partial}{\partial q_{i,w}} F{ (Q )} = \alpha \sum_{j} \beta_{ij} \frac{\partial \Omega_{ij}}{\partial q_{i,w}}
\end{equation}
use of these equations, when ADPs are not appropriate (like at cryogenic temperatures), greatly speeds up the computation.

\subsubsection{Periodic Boundary Conditions}
Periodic boundary conditions can be helpful when simulating extended solids or large nanoparticles. In this case all the non-crystallinity is contained within the simulation box and the box is repeated to create the longer distance peaks observed in the PDF. To perform this we can break up the Debye equation into two main parts, the part that describes the interatomic distances within the simulation box and those between boxes. Neglecting the thermal motion portion:
\begin{equation}
  F(Q) = \frac{1}{N \langle f \rangle^{2}}(\sum_{j\neq i} f_i^{*}(Q)f_j(Q) \frac{\sin(Qr_{ij})}{r_{ij}} + \sum_{i,j} f_i^{*}(Q)f_j(Q) \frac{\sin(QR_{ij})}{R_{ij}})
\end{equation}
where
\begin{eqnarray}
  R = |\vec{r} + \vec{\nu}|\\
  \vec{\nu} = \gamma_1*\vec{a} + \gamma_2*\vec{b} + \gamma_3*\vec{c}
\end{eqnarray}
where $\gamma_{i}$ is the number of copies of the simulation box in the $i$th direction, and $\vec{a}, \vec{b}, \vec{c}$ are the lattice or superlattice directions.

%This can be represented in the form of equation \ref{eq:abto} with ADPs as:
\begin{equation}
F(Q) = \alpha(\sum_{j\neq i}  \beta_{ij} \uptau_{ij} \Omega_{ij} + \sum_{i,j}  \beta_{ij} \zeta_{ij} \eta_{ij}) \label{eq:adp_pbc}
\end{equation}
where:
\begin{eqnarray}
    \zeta_{ij} = \exp(-\frac{1}{2}\varsigma_{ij}^{2}Q^{2}) \label{eq:zeta}\\
    \eta_{ij} = \frac{\sin(QR_{ij})}{R_{ij}} \label{eq:eta}\\
    \varsigma_{ij}^{2} = (\vec{u}_{ij} * \hat{D}_{ij})^{2}\\
    \hat{D}_{ij} = \frac{\vec{D}_{ij}}{R_{ij}}\\
    \vec{D}_{ij} = \vec{r} + \vec{\nu}
\end{eqnarray}
The additive nature of the first and second term of equation \ref{eq:adp_pbc} implies that we can separate the gradient.
Thus the gradient is:
\frac{\partial}{\partial q_{i,w}} F{ (Q )} = \alpha (\sum_{i!=j} \beta_{ij} (\frac{\partial \uptau_{ij}}{\partial q_{i,w}}  \Omega_{ij} + \uptau_{ij} \frac{\partial \Omega_{ij}}{\partial q_{i,w}}) + \sum_{i,j} \beta_{ij} (\frac{\partial \uptau_{ij}}{\partial q_{i,w}}  \Omega_{ij} + \uptau_{ij} \frac{\partial \Omega_{ij}}{\partial q_{i,w}})) \label{eq:pbc_grad_fq}
The first term gradient will be the same as equation \ref{eq:grad_fq}.
The second term however will have its own distinct gradients:
\begin{eqnarray}
  \frac{\partial \eta_{ij}}{\partial q_{i,w}}  = \frac{Q\cos(QR_{ij}) - \eta_{ij}}{R_{ij}^{2}} (q_{i,w}-q_{j,w} + \nu_{w})\\
  \frac{\partial \zeta_{ij}}{\partial q_{i,w}} = \frac{\varsigma_{ij}Q^{2} \eta_{ij}}{R_{ij}^{3}}   ((q_{i,w} - q_{j,w} +\nu_{w}) \varsigma_{ij}- (u_{i,w} - u_{j,w})R_{ij}^{2})\\
\end{eqnarray}
\begin{eqnarray}
\frac{\partial}{\partial u_{i,w}} F{ (Q )} = \alpha \sum_{j} \beta_{ij} \frac{\partial \uptau_{ij}}{\partial u_{i,w}}  \Omega_{ij}\\
\frac{\partial \uptau_{ij}}{\partial u_{i,w}} = - \frac{\sigma_{ij}Q^{2} \uptau_{ij}}{r_{ij}}  (q_{i,w} - q_{j,w})
\end{eqnarray}

\todo[inline]{Also should include PBC gradients, although they are trivial, maybe?}
\todo[inline]{How does this compare against the Ewald simulation technique for ionic solutions}

\begin{equation}
\frac{1.0 \sigma_ij}{R_{ij}^{3}} Q^{2} \uptau_ij \left(R_{ij}^{2} \left(- u_{ix} + u_{jx}\right) + \left(\nu_{x} + q_{ix} - q_{jx}\right) \left(\left(u_{ix} - u_{jx}\right) \left(\nu_{x} + q_{ix} - q_{jx}\right) + \left(u_{iy} - u_{jy}\right) \left(\nu_{y} + q_{iy} - q_{jy}\right) + \left(u_{iz} - u_{jz}\right) \left(\nu_{z} + q_{iz} - q_{jz}\right)\right)\right)
\end{equation}


