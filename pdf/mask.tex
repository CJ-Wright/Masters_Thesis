\subsubsection{Automated Mask Generation}
Detector masking is an important part of any x-ray scattering workflow as dead/hot pixels, streak errors, and beamstop associated features can be averaged into the data changing the signal and its statistical significance.
While some features, like the beamstop holder, can be easily observed and masked by hand other are much more difficult to observe even on large computer monitors.
Additionally, while dead/hot pixels and streaks are usually static the hot pixels associated with textured or single crystal scattering or cosmic rays are not.
Thus, coming up with an automated method for finding such erroneous pixels is important, especially as high flux diffraction beamlines can generate data very quickly.

While this problem can be quite complex in the most general case, we can use the annular symmetry of the powder scattering pattern to our advantage, by comparing a pixel against pixels in the same ring.
Since non-textured powder scattering should produce the same pixel intensity for a given ring we can mask any pixels which are $\alpha$ standard deviations away from the mean.
This method relies on the aforementioned pixel binning algorithm, as using miss sized bins will cause some pixels which should be in separate rings to be put together, and others which should be in the same ring to be separated.
In that case the masking algorithm will overestimate the number of pixels to be masked due to the additional statistical variation in the sample.

The masking algorithm procedure takes in the image and a description of the pixel positions in either distance from the point of incidence or in $Q$.
The image is then integrated twice, producing both the mean $I(Q)$ and the standard deviation of each $I(Q)$ ring.
The mask is created by comparing the pixel values against each ring's standard deviation and theshold $\alpha$.
Note that the theshold can be a funciton of distance from the point of incidence or $Q$.

To study the effectiveness of the masking we ran the algorithm against both simulated experimental data.
In the case of the simulated data four systems were created:
1) dead/hot pixels with varying numbers of defective pixels,
2) beamstop holder vith varying beamstop holder transmittance,
3) rotated beamstop holder vith varying beamstop holder transmittance,
and 4) beamstop holder with dead/hot pixels.
The base scattering was produced by
\begin{equation}
I = 100\cos(50r)^{2} + 150
\end{equation}
where $r$ is a pixel's distance from the beam point of incidence.
The positions of the dead/hot pixels were chosen at random as was the dead or hot nature of the defect.
The beamstop was positioned at the vertical center of the detector with an initial width of 60 pixels and final width of 120 pixels.
The hight of the beamstop was 1024 pixels.
The beamstop was calculated to attenuate the x-ray scattering signal at various transmittance, as various beamstop holder materials have different transmittance.



Figures \ref{fig:bs_10}-\ref{fig:dead_pixel_1000} show the results of the masking algorithm on simulated images.





Note that we do miss some pixels when the number of dead pixels grows very large.
However, most detectors do not have that many dead pixels.
We can run into these kinds of situations when running samples with some single crystal or textured components. 
However, when this is the case the contrast between the affected pixels and the desired signal is very large enabling easier masking.

Additionally the standard deviation threshold can be a function of the pixel distance from the center, allowing the mask generator to be more forgiving at certain points and stricter at others.
This is particularly helpful as the small number of pixels near the point of incidence combined with the very sharp peaks causes some pixels to be improperly masked.
Similarly it is important to remove dead pixels at the edge of the detector as these have an outsized effect on the integration as the pixel intensity is low to begin with.
In practice this results in the removal of almost all dead pixels and potentially the beamstop holder.
Removal of the holder depends on its individual properties, since a holder which is more x-ray opaque will cause a larger shift in the pixel intensity distribution.
The method was benchmarked on synthetic data, with both hot and cold pixels added.
Additional benchmarking was performed with synthetic beamstop holders of various x-ray transmittance.
Anomalous corner masking most likely due to the small number of pixels out at the corners.


\foreach \n in {10, 30, 50, 90}{
\begin{figure}
  \foreach \m in {raw, masked, missed}{
%     \includegraphics[width=.3\textwidth]{bad_bin_\m_\n}}
%     \foreach \m in {raw, masked, missed}{
    \subfloat[]{\includegraphics[width=.49\textwidth]{\m_\n}}
    }
%   \caption{Generated beamstop masks for a beamstop holder with $\n\%$ transmittance. The top figures are with a evenly binned detector, the bottom images are with bins at the detector $Q$ resolution. From left to right: the raw image, the masked image, and the missed pixels}
  \caption{Generated beamstop masks for a beamstop holder with $\n\%$ transmittance. a) the raw image, b) the masked image, c) and the missed pixels}
  \label{fig:bs_\n}
\end{figure}
}
\foreach \n in {100, 300, 500, 1000}{
\begin{figure}

  \centering
  \foreach \m in {masked}{
    \subfloat[]{\includegraphics[width=.5\textwidth]{bad_bin_dead_pixel_\m_\n}}
    \subfloat[]{\includegraphics[width=.5\textwidth]{dead_pixel_\m_\n}}
    }
\caption{Generated dead/hot pixel masks for a detector with $\n$ bad pixels. a) the poorly binned mask and b) the properly binned mask}
  \label{fig:dead_pixel_\n}
\end{figure}
}

ALSO NEED ROTATED BEAMSTOP MASKS

ALSO ALSO NEED TO SHOW THE EFFECT OF CHANGING ALPHA FOR THE MASKING

ALSO ALSO ALSO SHOW SOME MASKS OF REAL DATA, INCLUDING DATA WITH SINGLE CRYSTAL/TEXTURE