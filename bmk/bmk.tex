\graphicspath{{./bmk/figures/}}
\chapter{Benchmarking} \label{ch:bmk}
\todo[inline]{This entire section needs some rewritting to distinguish this from the paper}
\todo[inline]{Also some introduction would be great}
\todo[inline]{this just needs a lot of work}
\section{Introduction}
The NUTS-HMC system was tested on a series of nanoparticle (NP) benchmarks.
The purpose of these benchmarks is to test the ability of the NUTS-HMC system to reproduce the target PDF and its associated structure.
Systems were chosen for their size, crystallinity, and interfacial differences.

\section{PDF}
The formation of NPs with both crystallographic and non-crystallographic structures \cite{Marks1994} and with different chemical patterns \cite{Ferrando2008} are well documented.
For simplicity, we chose monometallic Au clusters as benchmarks and considered two groups of structures with different size and degrees of structural disorder in order to assess the reliability and efficiency of our HMC method for solving atomic structures from PDFs.
The first group consists of \ce{Au55} clusters with different degrees of disorder, including a crystalline cluster structure in $O_h$ (Octahedral) symmetry, a structure with a disordered surface, and an amorphous structure.
The second group consists of the crystallographically solved \ce{Au102} structure as in the \ce{Au102MBA44} nanocrystals \cite{Jadzinsky2007,Li2008}.
We used optimized structures from the Density Functional Theory (DFT) as target structures and generated the corresponding PDF, $G_{_\mathrm{obs}}$, according to
\begin{equation}
\label{Eq:Gdef}
  G_{_\mathrm{obs}} = \frac{2}{\pi} \int_{Q_\mathrm{min}}^{Q_\mathrm{max}} Q[S_{_\mathrm{obs}}(Q) - 1] \sin{\left (Q r \right )}\, dQ
\end{equation}
where $S_{_\mathrm{obs}}$ is the target structure's structure factor.
Since all the target structures were optimized by DFT at zero Kelvin the target and model PDF profiles were calculated at zero temperature, with no atomic displacement parameters (ADPs).
However, ADPs would have a considerable impact on the calculation of the PDF, especially for nanoparticles at non-zero temperatures.


Spin-polarized  DFT calculations were carried out using the Vienna ab initio simulation package (VASP) \cite{Kresse1993, Kresse1994} within the  Perdew-Burke-Ernzerhof (PBE) exchange-correlation functional \cite{PERD1996}.
The projected augmented wave method \cite{Blochl1994} and a kinetic energy cutoff of 400 eV were used.
Structural optimization was performed until the total energy and ionic forces were converged to 10$^{-6}$ eV and 10 meV/\AA, respectively.
The  amorphous \ce{Au55} structures were generated by simulated annealing using the classical embedded atom method potential \cite{Sheng2011}.
Different annealing temperatures between 1200 K and 1670 K (bulk melting temperature of Au) were used and the thermally equilibrated structures were cooled down to 300 K before minimization at 0 K.
Further optimization using DFT leads to total energies that vary within 1-2 eV among different amorphous structures  and the lowest energy one was used as the target structure. The target structure of \ce{Au102} was taken as the \ce{Au102} core of the DFT-optimized \ce{Au102MBA44} cluster \cite{Li2008}.

All systems were refined using a PES  which consists of a linear combination of $Rw$, the repulsive and attractive thresholded spring potentials.  The total potential energy in the Hamiltonian in Eq. (\ref{Hamiltonian}) is expressed as:
\begin{eqnarray}\label{eq:Ucomp}
  U(q) = U_{Rw}(q) + U_\mathrm{spring}(q, R_\mathrm{min}) + U_\mathrm{spring}(q, R_\mathrm{max})
\end{eqnarray}
The thresholded spring potentials are based on those previously proposed on by Peterson \cite{Peterson2014}, i.e.  $U_\mathrm{spring}(q, r_{t}) = \frac{\kappa}{2}\sum_{i, j}(r_{i, j}-r_t)^{2}$  for all atomic distance $r_{i,j}$  outside the bounds of the spring threshold $r_t$.
The resulting restoring forces on the out-of-bound atoms  bring the system back within the bounds of the PDF, $R_\mathrm{min}$ and $R_\mathrm{max}$, and therefore preventing the system from exploding or collapsing.
Otherwise,  incorrect refinements may result by having atomic pair distances out of the PDF bounds.  $\kappa$ is the spring constant in eV/\AA~and the $Rw$ potential is converted from unitless to eV via multiplication by a conversion factor $\lambda$.

Whereas the choice of the absolute values of $\lambda$ and $\kappa$ is somewhat arbitrary, their relative values are important in determining which  term in Eq.~(\ref{eq:Ucomp}) dominates the PES, especially when considering the effect of the simulation temperature.
Generally, the ratio between the total potential energy and the temperature determines how much random motion will dominate the dynamics; a lower ratio implies that random motion will play a large role in the dynamics.
The ratio between $\lambda$ and $\kappa$ of each spring describes how far the PDF can push the system below or above the bounds set by the spring potentials.
Heuristically, too stiff a spring  forbids the system to access new configurations, e.g.  high energy ``transition states'' which may involve shorter bonds or a larger system size.
Conversely, too small a spring constant makes it slower for the system to snap back within bounds and may lead to an explosion or implosion of the system, leaving the dynamics to drift aimlessly.

\subsection{Model Parameters}
Unless otherwise stated, the PDFs of the target and starting structures were generated using Eqn. (\ref{Eq:Gdef}) with a step of $\delta R=.01$~\AA, $Q_\mathrm{min}=0.1$~\AA$^{-1}$,  $Q_\mathrm{max}=25.0$~\AA$^{-1}$.
$R_\mathrm{min}$ and $R_\mathrm{max}$ correspond to the first minimum before the first PDF peak and that after the last PDF peak, respectively, which ensure that the full meaningful region of the PDF is modeled.  For each of the simulations, the Q resolution was calculated by
\begin{equation}
\delta Q=\frac{\pi} {R_\mathrm{max} + \frac{12 \pi}{Q_\mathrm{max}}}
\end{equation}

The HMC simulation was run with $N=300$  iterations, a target acceptance rate of 0.65, and an average starting momentum for each NUTS iteration  of 10 eVfs/\AA. Both  repulsive and attractive spring potentials are used with $\kappa=200$ eV/\AA ~and thresholds matching $R_\mathrm{max}$ and $R_\mathrm{min}$ of the PDF, respectively.  $\lambda=$ 300 eV was used as conversion factor for $Rw$. Each simulation was run with a pair of Nvidia GTX970 graphics cards, with one card partially occupied with desktop visualization.


\foreach \n in {au55sr, au55sd, au55a, au102tp, au147}{
    \input{bmk/\n}
}

\section{PDF with ADPs}
\foreach \n in {adp_50
%, adp_random, adp_janus
}{
    \input{bmk/\n}
}
\begin{enumerate}
\item Basic 50\% larger magnitude
\item Random addition to APDs
\item Janus ADPs
\end{enumerate}