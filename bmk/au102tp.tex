\subsection{Au102: triple phase}
Our final benchmark is \ce{Au102}, whose structure was initially solved by Jadzinsky and co workers using x-ray crystallography \cite{Jadzinsky2007} and further confirmed by DFT studies \cite{Li2008}. The \ce{Au102} structure consists of three main parts, a 49-atom Marks decahedron core, two $C_{5}$ caps consisting of 20 atoms each, and 13 equatorial atoms. Unlike previous cases, the multi-symmetry nature of the structure, i.e. each part has its own distinct symmetry,  poses a  challenge for PDF-based solution of the structure. This is because of the atomically centralized nature of the PDF, in which each atom ``sees'' a density of other atoms surrounding it and has a strong tendency towards becoming the center of the main symmetry group. Such tendency may lead to a solution where some of the correct atomic symmetries are discarded in favor of the core symmetry.

\subsubsection{ Starting from FCC structure}
The starting structure was generated by a spherical cut of the FCC bulk lattice, with two surface atoms removed to conserve the total number of Au atoms.

$R_\mathrm{min}$ and $R_\mathrm{max}$ for this simulation were 2.7 \AA ~and 16. \AA, respectively, with $\delta Q=0.18$~\AA$^{-1}$. The simulation ran for approximately two hours, over a total of $\sim$82 thousand configurations.
The results of the simulation are shown in Fig.~\ref{fig:Au102_fcc}.

\begin{figure}[!h]
  \def \localimgpath {Au_102_HMC_paper_final/55d7b3d4d2d355710ddb3fdc}
  \centering
  \foreach \m in {pdf, coord, rbonds, angle, target, min}{
     \subfloat[]{\includegraphics[width=.47\textwidth]{\localimgpath_\m}}\quad
     }
     \caption{Similar to figure \ref{fig:Au55_sd} for  \ce{Au102} as in DFT-optimized \ce{Au102MBA44} cluster.}
     \label{fig:Au102_fcc}
\end{figure}

The initial structure of an fcc bulk-cut cluster, had a starting $Rw$ of 77.6\% (see Fig. S4), whereas the final structure has a $Rw$ as low as 8.1\%.
The disagreement between the final and target PDFs shows that the majority of the error is in the high $R$ region, which is related to the long range distances between the core, caps, and equatorial atoms.
The agreement for other structural metrics is less satisfactory. The bond angle distribution for core atoms in the final structure has a poor correlation with those in the target structure, with much broader peak widths.
This is likely caused by the high kinetic barrier to change from one high-symmetry core structure (fcc) to another (Marks Decahedron). In contrast,
the bond angle distribution for surface atoms, which are of lower symmetry than the core, show a much better agreement.
This is due to the preference of Monte Carlo techniques for higher entropy, and thus lower symmetry, structures.
Similarly, the radial bond distance does not show the correct clustering of bond lengths as expected from an ordered structure, indicating the amorphous nature of our fit.
Finally, the CN distribution shows the largest discrepancy at CN=12, again due to the amorphous nature of the fit. Overall, the structural metrics beyond the PDF indicate the poor agreement between the final and target structures.
A higher simulation temperature, potentially combined with CN or bond length aware potentials (such as DFT or EXAFS derived PESs) may help to resolve this discrepancy.

\subsubsection{Marks decahedron}
The starting structure, a Marks Decahedron, was generated by the ASE Cluster tool with 2  atoms on the [100] face normal to the 5-fold axis, 3 atoms on the [100] plane parallel to the 5-fold axis and, 1 atom deep Marks re-entrance.  This produced a structure with 101 atoms which was extended by one more Au atom to fill out the \ce{Au102} structure.

$R$ bounds and Q resolution were the same as the previous case. The simulation ran for approximately 2.5 hours over a total of $\sim$90 thousand configurations.  The results of the simulation are shown in Fig.~\ref{fig:Au102_deca}.

\begin{figure}[!h]
  \def \localimgpath {Au_102__deca_HMC_paper_final/55db6901d2d3552df4f6c86f}
  \centering
  \foreach \m in {pdf, coord, rbonds, angle, target, min}{
     \subfloat[]{\includegraphics[width=.47\textwidth]{\localimgpath_\m}}\quad
     }
    \caption{Similar to Fig.~\ref{fig:Au102_fcc} with Marks decahedron as the starting structure.}
     \label{fig:Au102_}
\end{figure}

The starting structure of Marks decahedron ($Rw$=56.6\% , see Fig. S5) yielded a better structural solution, with a final $Rw$ of 3.3\%.
However, the discrepancies at high $R$ remains as in the previous case. By examining the final structure, we can see that these high $R$ errors are due to a lack of the two 20-atom caps and 13 equatorial atoms. Similarly, the radial bond distance distribution displays a diffusive behavior unlike the bond length clustering in the target structure.
Compared to the previous case, the agreement in the CN and bond angle distributions are improved, with the latter  capturing nearly all peaks in the target structure with the exception of the 110\degree ~bond angle.
Relatively large discrepancies are found in the CN distribution at the low and high ends.