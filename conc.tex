\chapter*{Conclusions}
The work here presents one of the most complete end to end approaches to processing, analyzing, and simulating atomic pair distribution function data.
The goals of this work were to build a modular, quick, and robust method for handeling both experimental PDF data and solving atomic structures from said data.

The statistical mechanical PES solvers were designed to robustly find atomic solutions which are global minima of the PES.
This was accomplished by using some of the most advanced Monte Carlo algorithms and samplers.
The analytical equations for the PES and its gradients were derived to provide the quickest searches.

The PDF gradients were derived and implemented as GPU kernels to furthur speed up the PES search.
The inclusion of the GPUs, combined with the atom pair mapping, were found to provide a 10x to 100x speedup over a multiprocessed CPU methodology.

The extensive benchmarking of the NUTS-HMC system presented in chapters \ref{ch:pes_e} and \ref{ch:pdf} showcased the system's robustness, speed, and effectiveness.
Interestingly it seems the the simulations also helped to elucidate the relationship between $Rw$ and the resulting fit of the secondary metrics, including radial bond distribution.
This is particularly important as it begins to establish $Rw$ goals and a relationship between $Rw$ and the confidence that features from the underlying structure that the PDF represents are reproduced by the structural model which is produced my Monte Carlo modeling.
It seems that the theshold for acceptable $Rw$ in Monte Carlo modeling needs to be quite lower than the current literature standards to properly reproduce the structure.

A novel data processing workflow was also developed