\chapter{Introduction} \label{intro}
Engineering materials and chemicals on the atomic scale has long been a goal for the chemistry, physics, materials science, and chemical engineering fields.
Realizing this goal could lead to durable fuel cell catalysts, more bioavailable pharmaceuticals, and radiation damage resistant spacecraft shielding.
Before we can even think of making atomistically exact structures, durable structures, or structures which change in reproducible ways, we need to know the atomic structure exactly.
This work addresses these issues by developing a methodology for solving the structure of nanomaterials by matching experimental x-ray scattering data with simulated atomic structures.

Chapter \ref{ch:pes_e} develops the statistical mechanical system used to match the theoretical structure.
\S \ref{sec:pes} focuses on the development of potential energy surfaces, including potential energy and force equations, which have minima where experimental results and simulated structures agree the most.
\S \ref{sec:ens} will discuss statistical mechanical ensembles which are used to search for minima on the potential energy surface.

Chapter \ref{ch:pdf} will discuss the mathematical and computational development of the atomic pair distribution function (PDF).
\S \ref{sec:comp} will focus on the rapid graphical processing unit based calculation of the PDF and its gradients.

Chapter \ref{ch:bmk} will discuss the benchmarking of the the combined statistical mechanical optimizer and PDF calculation systems against a series of theoretical nanoparticles, focusing on understanding limitations of the method and structure reproduction.

Chapter \ref{ch:dp} will focus on the acquisition of experimental data, its management, and processing.
\S \ref{subsec:qres}, \ref{subsec:mask}, and \ref{subsec:int} will discuss the derivation of the $Q$ resolution function, the automated masking of 2D area detectors for x-ray total scattering measurements using the previously derived $Q$ resolution, and the impact of different averaging methods and masks on azimuthal integration, respectively.

Chapter \ref{ch:pno} will discuss preliminary experimental results investigating the phase changes and local structure of \ce{Pr2NiO4}, revealing the influence of thermal history on the structure.