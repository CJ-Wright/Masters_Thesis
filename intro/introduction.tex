\chapter{Introduction} \label{intro}
Engineering materials and chemicals on the atomic scale has long been a goal for the chemistry, physics, materials science, and chemical engineering fields.
Realizing this goal could lead to more durable fuel cell catalysts, bioavailable pharmaceuticals, and radiation  resistant shielding.
Before we can even think of making atomistically exact, durable, or reproducibly changing structures, we need to know the precise atomic structure.
This work bridges the gap in structural knowledge by developing a methodology for solving the structure of materials by matching experimental x-ray total scattering data with simulated atomic structures.

Chapter \ref{ch:pes_e} develops the statistical mechanical system used to match experimental and theoretical structures.
\S \ref{sec:pes} focuses on potential energy surfaces, including potential energy and force equations, which have minima where experimental results and simulated structures agree the most.
\S \ref{sec:ens} will discuss the statistical mechanical ensembles which search the potential energy surface for minima.

Chapter \ref{ch:pdf} develops the mathematical and computational framework for the atomic pair distribution function (PDF).
\S \ref{sec:comp} will focus on the rapid graphical processing unit based calculation of the PDF and its gradients.

Chapter \ref{ch:bmk} will discuss the benchmarking of the the combined statistical mechanical optimizer and PDF calculation systems against a series of theoretical nanoparticles.
These benchmarks will focus on understanding the limitations of the method and the relationship between goodness of fit and structure reproduction.

Chapter \ref{ch:dp} will focus on the acquisition of experimental data, their management, and processing.
\S \ref{subsec:qres}, \ref{subsec:mask}, and \ref{subsec:int} will discuss the derivation of the $Q$ resolution function, the automated masking of 2D area detectors using the previously derived $Q$ resolution, and the impact of different averaging methods and masks on azimuthal integration, respectively.

Chapter \ref{ch:pno} will discuss preliminary experimental results investigating the phase changes and local structure of \ce{Pr2NiO4}, revealing the influence of thermal history on the structure.
This chapter will also analyze the discrepancy between the reciprocal space scattering and the PDF.